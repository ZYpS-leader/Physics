\chapter{近代物理学初步}
\CTEXsetup[format+={\Large}]{section} 
\section{近代物理那些事}
\subsection{近代物理的发展史}
\paragraph{近代物理是从什么时候开始的?}\ 
\\1897年的Thomson发现电子以来,自此拉开了近代物理学的序幕.包括此前伦琴所发现的\textbf{X射线}和贝克勒尔发现的\textbf{物质放射性},以及之后普朗克、爱因斯坦、薛定谔、波尔、泡利和古兹密特等人的成果并称为旧量子学.
\paragraph{从微观到宏观,究竟是什么样的进步?}\ 
\\从以牛顿力学为基础的经典力学出发,我们有了宏观物理学,也就是所谓\textbf{经典物理学}.而自近代以来,科学的飞速进步让人们有了进入微观世界能力.从此我们可以更深入的观察微观粒子间的关系,去探究神秘的微观世界.那就让我们带着对以上各位近代物理学家的敬仰,一同走进近代物理学的世界! 
\subsection{什么是近代物理}
\textbf{近代物理}\cite{28}主要研究的是量子力学和相对论
\section{微观模型:让我们再深入些}
\section{量子力学初步}
\subsection{什么是量子力学}
\subsection{泡利不相容}
\subsection{物质波}
\subsection{波粒二象性}
\subsection{量子纠缠}
\section{第6章例题}
\section{第6章习题}