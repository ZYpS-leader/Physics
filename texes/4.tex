\section{能量学定律}
在前几节中,我们讲了\textbf{运动学}、\textbf{牛顿三定律}和\textbf{力的合成与分解}.它们是经典物理学的重要组成部分,而为了进一步加强它们间的联系,我们进一步引入\textbf{能量}的概念.
\subsection{能量守恒定律}
什么是\textbf{能量}?所谓\textbf{能量}其实就是一个物体(或系统)所能做功的本领,用$E$表示,国际单位J,且$\dim J=kg\cdot m^2/s^2$.($\dim$是什么意思?参见\textbf{\ref{lg}一节.})此外还有单位如电子伏特ev等单位.电子伏特\textsf{ev}将在\textbf{\ref{dsn}}中提到.


回到能量守恒定律.我们已经了解了什么是\textbf{能量},那\textbf{能量}的大小和公式是什么?它就是大家耳熟能详的质能方程$E=mc^2$.\textbf{注意!它仅建立在狭义相对论上!}那它满足什么条件呢?


在大自然中所有物体的能力总和均保持不变.在任何情况下\textbf{都}保持不变!也可以仅考虑一个系统中的能量之和守恒,但此时该系统必须\textbf{孤立}外部,避免能量增加或损失.
\subsection{动能定理}\label{dndl}
还是先来回顾下什么是\textbf{动能}.动能$E_k=\frac12mv^2$,文字表达为一物体的动能与该物体质量和速度的平方成正比.看到刚刚讲过的\textbf{能量守恒定律},该物体哪怕不孤立于外界,但是它所损失的势能即为它增加的动能,表达式为$\Delta E_k=\Sigma W$,即动能变化量等于合外力做功.该等式在任何条件下成立,无需前提. 
\subsection{机械能守恒}
什么是\textbf{机械能}?机械能为动能$E_k$和势能$E_p$(仅包括重力势能和弹性势能).在\textbf{\ref{dndl}}中,我们知道动能的变化量是合外力做功,而势能的改变必有外力做功.与此同时,外力$F$做功意味着物体将有速度$v$的变化.这两个力必定相反.于是当且仅当动能 的变化量刚好等于势能变化量的相反数时,机械能守恒.而此时受到两个外力的矢量和为0,故合外力为0.


因此机械能守恒公式表达为$ E=E^{'}$  ,iff $F_{\mbox{\scriptsize 合}} =0$. 
\subsection{动量和冲量定律}
\subsubsection{动量和冲量}
什么是\textbf{动量}?动量是描述一个物体运动能力的物理量,用符号$p$表示,对任意物体有$p=mv$,其中$v$为物体当前的运动速度.


什么是\textbf{冲量}?冲量是描述一个力所做功多少的量,注意与功$W$区分.冲量用符号$I$表示,对力有$I=Ft$,$t$为该力做功时间.
\subsubsection{动量守恒}
当一个物体所受合外力为0时,它的运动状态不会发生改变,即速度$v=v^{'}$.如果该物体质量守恒,那么显然有$p=p^{'},mv_1=mv_2$.如果它质量改变呢?那根据牛顿第二定律显然加速度减小,速度减小.如果只是单纯的改变质量,那速度应当恰好和质量成反比,仍然有$p=p^{'},mv_1=mv_2$.
\subsubsection{动量定律}
要改变物体的动量$p$,就需要一个外力来改变它的速度.我们会自然的想到冲量$I$.那$p$和$I$之间有什么关系呢?


我们不妨做一下计算.先假设物体质量不变,那么有 
\begin{align}
&\Delta p_{1\to 2}(\mbox{即1状态到2状态的动量变化量})\notag 
\\={}& m_1v_1-m_2v_2(\mbox{拆分两个动量})\notag 
\\={}&m\Delta v(\mbox{由于质量相等,提取公因式m})\notag 
\\={}&m(a\Delta t_{1\to 2})(\mbox{由运动学公式$\Delta v=a\Delta t$})\notag 
\\={}&(ma)t(\mbox{重新组合三个量})\notag 
\\={}&Ft=I(\mbox{由牛二有$F=ma$})
\end{align}
因此有$\Delta p=I$.
\subsection{**其他能量**}
在此节中简要介绍一下能量$E$的单位.
\paragraph{1.J}\qquad 这是能量的国际单位,$1 \textsf{J}=$\textsf{$kg\cdot m^2/s^2$}.
\paragraph{2.ev}\qquad 电子伏特,表示一个单位元电荷(带电量为$1.6\times 10^{-19}\textsf{C}$)通过一个单位电场(电压为1V的电场)所获得的的动能,$1\ \textsf{ev}=1.6\times 10^{-19}\textsf{J}$.
\paragraph{3.cal}\qquad 卡路里,表示1克水(标准水,即密度恰好为$1\times 10^3kg/m^3$)在一个标准大气压下升高一度所需要的热量.$1\ \textsf{cal} =4.19J $.

\paragraph{4.erg}\qquad 尔格.这个单位非常非常少用,在此仅做补充.一尔格的能量能使一达因的力让物体在力的方向上移动1厘米.一达因的力可以使一个质量是1克的物体产生$1\times 10^{-2}m/s^2$的加速度.
\subsection{1.4例题}
\subsection{1.4习题}