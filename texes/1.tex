 \chapter{运动学和力学}
 \section{运动学}
 运动学是物理四大基础学科之一,在各方面都起到了非常重要的奠基作用.因此,在第一章中,我们将一起探索运动的规律.
 \subsection{机械运动}
 首先,我们需要了解什么是\textcolor{red}{运动}.在物理中,运动可以分成\textcolor{red}{ 机械运动}和\textcolor{red}{非机械运动}.本章节中,主要以机械运动为核心展开.非机械运动,即波动,会在后续章节中讲到.
 \subsubsection{机械运动的定义和形式} 
 \subsubsubsection{机械运动的定义}
 \paragraph{机械运动} \ \ \ 从物理角度出发,包含了绝大多数宏观上的运动,但不包括大部分波动.机械运动需要选取合适的参考系(即惯性系)来分析问题.在\textcolor{red}{惯性系}中,惯性力也可以在受力分析时作为力出现,但此时的参考系不固定.我们会在第2章中详细讲到相关内容!
 \subsubsubsection{机械运动的形式}
 \paragraph{机械运动}\ \ \ 即一个物体相对于参考系中心(简单点说,另一个物体)产生了位移.通过计算加速度a、速度v、路程S等分析其运动状态,这就是本章的研究对象.常见的单物体机械运动包括:直线运动和曲线运动,又根据加速度不同分为匀速、匀加减速等.
 \subsection{匀速直线运动}
 此类运动问题更多的出现在数学中.由于它"匀速",因此$v_0=v_t$,且有$S=vt$,$a=0$两个结论.由于速度的不同和起始点的不同,出现了追及问题、相遇问题等,但它们一般较为简单(复杂的情况会在例题中进行讲解),这里不再展开.
 \subsection{匀加减速直线运动}
 在匀加减速直线运动中中,速度随时间的增加而增加,而单位时间的增加量叫做加速度a.速度和时间成线性变化,而路程是速度对时间的积分.通过以上定义不难得出有$S=\frac{1}{2}at^2+v_0t$和$v_t=v_0+at$.读者可以尝试自己证明一条引申公式:$\Delta S=\Delta(v^2)$或,$v^2_t-v^2_0=2as$.并不复杂!
 \subsection{平抛运动}
 抛体运动可将运动拆分成水平方向和竖直方向.
 在平抛运动中,水平方向分速度恒定,始终等于抛出时速度.与此同时在竖直方向上做自由落体运动,即加速度为g的匀加速运动.于是有$v_xt=v_0$,$v_yt=gt$,$S_xt=v_0t$,$S_yt=\frac{1}{2}gt^2$.思考:$\vec{S_t}$是多少?
 \subsection{斜抛运动}
 斜抛运动指以与地面有夹角$\theta$的初速度$v_0$抛出.不难发现竖直方向它的加速度仍然为g,但是先"上抛"再"下抛".先将初速度按方向分解,有$v_x=v_0\cos\theta$,$v_{y0}=v_0\sin\theta$.$v_x$保持不变,而$v_y$有向下的加速度g,即$v_{yt}=v_0\sin\theta-gt$.类似平抛运动,我们计算两个分方向的路程:$S_x=v_0t\cos\theta$,$S_y=v_0t\sin\theta-\frac{1}{2}gt^2$.不难看出,斜抛运动和平抛运动在水平分量上一致,即在水平方向分动能守恒.为什么呢?请先自己思考,我们将在第四章节中证明这个结论.
 \subsection{1.1例题}
 \paragraph{例1} 
 \ \ \ \ 两相同的小球从同一点以相同的初速度$v_0$斜向上抛出,A小球的初速度与水平地面夹角为$\alpha$,B小球的初速度与水平地面夹角为$\beta$.A小球经过时间$t_1$后落于地面上的C点.一定时间后B小球也落于C点.设起始点与C点的距离为$l$,则:\\(1)计算$l$;\\(2)$\alpha$和$\beta$的关系是?
 \paragraph{解}
 \ \ \ \ \  首先,设B小球运动时间为$t_2$,然后通过斜抛运动的公式有:$v_0\sin\alpha-t_1g=0,v_0\sin\beta-t_2g=0,v_0\cos\alpha t_1=l=v_0\cos\beta t_2$,那么第一问答案显而易见:$l=v_0\cos\alpha t_1$.然后联立两组等式,得$\sin\alpha=\frac{t_1g}{v_0},\sin\beta=\frac{t_2g}{v_0}$,即$\frac{\sin\alpha}{\sin\beta}=\frac{t_1}{t_2}$,同时$\frac{\cos\alpha}{\cos\beta}=\frac{t_2}{t_1}$,相乘得$\sin\alpha\cos\alpha=\sin\beta\cos\beta$,因此得解$\alpha=\beta$或$\alpha+\beta=0$或$\alpha=\pm(\pi/2\pm \beta)$.前两个解显然不合理,而$\alpha$和$\beta$都在开区间$(0,\pi/2)$中,因此最终得到答案$\alpha+\beta=\pi/2$.\\\ \ \ \ 这道题考验的是对斜抛运动两个公式的熟练运用,同时对数学公式有一定要求.本题中解的三角方程是典型的反三角方程,此类方程在物理中非常常用.同时还需要熟练掌握和差化积、积化和差、辅助角公式等.它们往往非常有用!
 \paragraph{例2} 
 \ \ \ \ 
 在水平平面上有两点点A、B.已知A点在B点左侧.小球$p_1$从B点以垂直于平面的速度$v_1$向上抛出.同时A点的小球$p_2$以与水平面夹角$\theta$的初速度$v_2$抛出(初速度向右).两球在时间t后相撞并停止运动.相撞点距离平面高度为h.求h的所有可能值和此时$\theta$需要满足的条件.
 \paragraph{解}
 \ \ \ \ 本题考察多解问题.有四种可能,分两类:$p_2$球速度向上时/向下时.每一类又分为两种可能:$p_1$球向上/向下.接下来我们分类讨论.\\
 \textbf{1:都在向上运动.}带入匀加速运动公式有$2gh=v^2_t=(gt)^2$,得到$h=\frac{1}{2}gt^2$.再带回$p_2$的斜抛公式,有$S_y=h=\frac{1}{2}gt^2=v_2t\sin\theta-\frac{1}{2}gt^2$,此时$\sin\theta=\frac{gt}{v_2},\theta=\arcsin\frac{gt}{v_2}$.\\
 \textbf{2.$p_1$向上,$p_2$向下.}从$p_1$的运动状态来看,仍然有$h=\frac{1}{2}gt^2$,但s$p_2$的状态略有改变.诶?好像还是$\theta=\arcsin\frac{gt}{v_2}$?对不对?有什么问题?尝试自己解答这个问题!\\
 \textbf{3.$p_1$向下,$p_2$向上.}$p_2$的高度还是$=v_2t\sin\theta-\frac{1}{2}gt^2$,但$p_1$的状态发生了改变.我们先计算$p_1$何时速度降低到0?很显然,$t_0=\frac{v_1}{g},S_1=\frac{v^2_1}{2g}$.此后再做匀加速运动时长$\frac{gt-v_1}{g}$,末态速度$gt-v_1$.路程$\Delta h=\frac{(gt-v_1)^2}{2g}$.然后带入$p_2$的斜抛,$v_2t\sin\theta-\frac{1}{2}gt^2=\frac{v^2_1-(gt-v_1)^2}{2g}=\frac{2v_1t-gt^2}{2}$.因此$\sin\theta=\frac{v_1}{v_2},\theta=\arcsin\frac{v_1}{v_2}$.\\
 \textbf{4.都在向下运动.}与第3种情况仅有略微的差异,不妨自己试试?\\\\
 {\CJKfamily{zhkai} 其实本题有些错误的地方,非常明显...试着把它们找出来并修正!}\\
{\CJKfamily{zhkai} 修正:}\_\_\_\_\_\_\_\_\_\_\_\_\_\_\_\_\_\_\_\_\_\_\_\_\_\_\_\_\_\_\_\_\_\_\_\_\_\_\_\_\_\_\_\_\_\_\_\_\_\_\_\_\_\_\_\_\_\_\_\_\_\_\_\_\_\_\_\_\_\_\_\_\_\_\_\_\_\_\_\_\_\_\_\_\_\_\_\_\_\_\_\_\_\_\_\_\_\_\_\_\_\_\_\_\_\_\_\_\_\_\_\_\_\_\_\_\_\_\_\_\\\_\_\_\_\_\_\_\_\_\_\_\_\_\_\_\_\_\_\_\_\_\_\_\_\_\_\_\_\_\_\_\_\_\_\_\_\_\_\_\_\_\_\_\_\_\_\_\_\_\_\_\_\_\_\_\_\_\_\_\_\_\_\_\_\_\_\_\_\_\_\_\_\_\_\_\_\_\_\_\_\_\_\_\_\_\_\_\_\_\_\_\_\_\_\_\_\_\_\_\_\_\_\_\_\_\_\_\_\_\_\_\_\_\_\_\_\_\_\_\_\_\_\_\_\_\_\_\\\_\_\_\_\_\_\_\_\_\_\_\_\_\_\_\_\_\_\_\_\_\_\_\_\_\_\_\_\_\_\_\_\_\_\_\_\_\_\_\_\_\_\_\_\_\_\_\_\_\_\_\_\_\_\_\_\_\_\_\_\_\_\_\_\_\_\_\_\_\_\_\_\_\_\_\_\_\_\_\_\_\_\_\_\_\_\_\_\_\_\_\_\_\_\_\_\_\_\_\_\_\_\_\_\_\_\_\_\_\_\_\_\_\_\_\_\_\_\_\_\_\_\_\_\_\_\_\\\_\_\_\_\_\_\_\_\_\_\_\_\_\_\_\_\_\_\_\_\_\_\_\_\_\_\_\_\_\_\_\_\_\_\_\_\_\_\_\_\_\_\_\_\_\_\_\_\_\_\_\_\_\_\_\_\_\_\_\_\_\_\_\_\_\_\_\_\_\_\_\_\_\_\_\_\_\_\_\_\_\_\_\_\_\_\_\_\_\_\_\_\_\_\_\_\_\_\_\_\_\_\_\_\_\_\_\_\_\_\_\_\_\_\_\_\_\_\_\_\_\_\_\_\_\_\_\\\_\_\_\_\_\_\_\_\_\_\_\_\_\_\_\_\_\_\_\_\_\_\_\_\_\_\_\_\_\_\_\_\_\_\_\_\_\_\_\_\_\_\_\_\_\_\_\_\_\_\_\_\_\_\_\_\_\_\_\_\_\_\_\_\_\_\_\_\_\_\_\_\_\_\_\_\_\_\_\_\_\_\_\_\_\_\_\_\_\_\_\_\_\_\_\_\_\_\_\_\_\_\_\_\_\_\_\_\_\_\_\_\_\_\_\_\_\_\_\_\_\_\_\_\_\_\_\\\_\_\_\_\_\_\_\_\_\_\_\_\_\_\_\_\_\_\_\_\_\_\_\_\_\_\_\_\_\_\_\_\_\_\_\_\_\_\_\_\_\_\_\_\_\_\_\_\_\_\_\_\_\_\_\_\_\_\_\_\_\_\_\_\_\_\_\_\_\_\_\_\_\_\_\_\_\_\_\_\_\_\_\_\_\_\_\_\_\_\_\_\_\_\_\_\_\_\_\_\_\_\_\_\_\_\_\_\_\_\_\_\_\_\_\_\_\_\_\_\_\_\_\_\_\_\_\\
 \subsection{1.1习题}
 \paragraph{}\ \ \ \ 本章的习题非常简单:证明1-6节中让你尝试证明的部分!