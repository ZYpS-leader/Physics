\section{牛顿三定律}
\subsection{牛顿第一定律:惯性}\label{gx}
\subsubsection{探究:惯性和什么有关?}
在初中物理中,我们已经学习了什么是\textbf{惯性}.那惯性有什么有关?有什么关?想必大家心中其实都有答案.本节就让我们再次走进惯性,看清其背后的本质!


先让我们回顾一下初中的惯性实验吧:我们将三个不同质量的小木块分别以相同初速度$v_0$滑上毛毯、木板和玻璃,比较它们停下来的位置.通过对比,我们发现:\textcircled{1}物体质量越大,停下来的越早;\textcircled{2}地面粗糙程度越大,物体停下来的越早.这意味着物体的\textbf{惯性}与它的质量和地面粗糙程度有关.所以\textbf{惯性}究竟是什么?揭晓答案:一个表述使物体运动状态发生改变难易程度的物理量.至于为什么和上述两个物理量有关,我们将在下一节中说明.
\subsection{牛顿第二定律:加速度和力的关系}
\subsubsection{$F=ma$}
从标题就知道,\textbf{牛顿第二定律}的表达式是$F=ma$.这意味着物体加速度与方向上受力成正比、与质量成反比.有眼尖的同学可能就要问了:等式左边的单位是牛(N),但右边却是千克·米/秒$^2$,这不是违背常理了吗?!我知道你很急,但你先别急,看看第\ref{lg}节吧!
\subsubsection{非惯性系}
在第\ref{gx}一节中,我们讲到了什么是惯性.但它并不是像力那样真实存在的东西,所以我们要引入一个\textbf{非惯性系}来描述它.别看标题以为这个系里没有惯性!它研究的正是惯性.在非惯性系中,将本来不可见的惯性具象化为一个力:$\boldsymbol{\mathrm{F}_v}$,它满足牛顿第二定律,只不过其中的速度$v$是相对于参考系中心的相对速度.因此,非惯性系在研究连接体问题等多物体运动时非常有用.
\subsubsection{再谈惯性}
在\ref{gx}一节中,我们留下来一个问题:\emph{为什么惯性和物体在方向上的速度无关,但和粗糙程度和质量有关?}在学过牛顿第二定律后,相信你能自己解决这个问题.事实上,由于在水平方向上物体的运动只受摩擦力$f$影响,因此$f$的大小决定了它的惯性.由公式可知,$f=\mu N=ma$,因此$a=\frac{\mu N}{m}=\mu g$,因此惯性和物体质量、粗糙程度有关.

\subsection{牛顿第三定律:作用力与反作用力}
\subsubsection{证明:作用力与反作用力矢量和为0}
牛顿第三定律告诉我们,如果两个物体有接触并相互作用,保持平衡(即一起做匀速直线运动或同时静止),那么它们间的作用力等于对方的反作用力.很好证明:在\textbf{定理\ref{dl1.1}}中,我们知道它们间的一对反作用力共线;而他们保持平衡,因此合力为0.
\subsection{**量纲**}\label{lg}
\subsubsection{什么是量纲}
我们知道,在国际单位制中,有这7个基本单位:\footnote{本节中标点符号不同于前文,它们是中文(全角)标点,但意思没有任何区别.不过是不同作者的习惯不同!——审核注}
\begin{equation*}
    \mathrm{m}\qquad\mathrm{kg}\qquad\mathrm{K}\qquad\mathrm{s}\qquad\mathrm{cd}\qquad\mathrm{mol}\qquad\mathrm{A}
\end{equation*}

它们分别代表长度、质量、温度、时间、光强、物质的量、电流的单位。而生活中或学术中的各种单位,都可以由这七个基本单位通过乘法、除法、乘幂的组合导出,它们就是导出单位\footnote{平面角度和立体角度例外,你可以将它视为无单位数——作者注},如$\mathrm{J}$、$\mathrm{V}$、$\mathrm{Pa}$等。为了表明物理量之间的关系,人们引入了\textbf{量纲}这一概念。

长度、质量、温度、时间、光强、物质的量、电流的量纲分别表示为\(L\  M\  \Theta\  T\  J\  n\  I\);而一个物理量$X$的量纲用\(\dim X\)来表示,简写为$[X]$。比如
\[
    \dim v = LT^{-1}\qquad \qquad \dim W = ML^2T^{-2}
\]

\subsubsection{量纲的应用}
\emph{量纲有什么作用呢?}

想象一个场景:你在一次物理考试中,有一道题需要用到开普勒第三定律
\[
    T = 2\uppi \sqrt{\frac{R^3}{\mathrm{G}M}}
\]
但你死活背不出来。你只记得系数是$2\uppi$。在这种情况下,你就可以运用到量纲分析法了。发挥你的头脑风暴,思考:天体的运动周期和哪些量有关呢?首先,天体受到引力的吸引,因此和万有引力常数\(\mathrm{G}\)有关,还和母星质量有关(这里我们假设行星质量$\ll$母星质量)。再根据开普勒第三定律的文字表述(不会有人连这个都背不下来吧)得到这和行星到母星的距离(这里假设行星质量忽略不计,也没有初速度,所以行星以匀速圆轨道运行)有关。综上,我们不妨设
\[
    T = 2\uppi R^\alpha G^\beta M^\gamma
\]
由于\(\dim\mathrm{G} = L^3M^{-1}S^{-2}\qquad\dim R = L\qquad\dim M = M\qquad\dim T=T\),故
\[
    \left\{\begin{lgathered}
        3\beta+\alpha=0\\
        -\beta+\gamma=0\\
        -2\beta=1
    \end{lgathered}\right.
\]
可解出未知数,再代入原式中,公式也就出来了。
\subsection{1.3例题}
\subsection{1.3习题}